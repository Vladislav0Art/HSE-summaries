\Subsection{Пространства Лебега}

\begin{definition}
    $\mu$ -- мера, $p \geq 1$.

    $L^{p} (E, \mu) := \{f : E \rightarrow \bar{\mathbb{R}}$ - измеримые, т.ч. $\int_{E} |f|^p d\mu < \infty\}$  -- векторное пр-во.

    $|| f ||_p = \left( \int_{E} |f|^p d\mu \right)^{\frac{1}{p}}$

    \begin{enumerate}
        \item Нер-во треугольника -- нер-во Минковского: $||f+g||_p \leq ||f||_p + ||g||_p$.
        \item Неотрицательность: $|| f ||_{p} \geq 0$.
        \item Константа выносится: $|| \alpha \cdot f ||_p = |\alpha| \cdot ||f||_p$.
        \item Но в нуле не всегда значение равно нулю: $||f||_p = 0 \Rightarrow$ интеграл от неотрицательной функции $|f|^p$ равен нулю $\Rightarrow$ $f = 0$ \textbf{почти везде}.
    \end{enumerate}

    Рассматриваем не функции, а классы эквивалентности с точностью до совпадения почти везде.

    Проблема: нет значения функции в точке.
\end{definition}

\begin{definition}
    \textbf{Существенный супремум} ($esssup$ или $rraisup$)

    $a$ -- существенный супремум функции $f: E \rightarrow \bar{\mathbb{R}}$,
    
    если $a = \inf \{ c \in \mathbb{R} : f(x) \leq c \text{ при почти всех $x \in E$}\}$.
\end{definition}

\begin{properties}
    \begin{enumerate}
        \item $esssup f \leq \sup f$
        \item {
            $f(x) \leq esssup f$ при почти всех $x$

            \begin{proof}
                $a := esssup f \implies \exists e_n : \ \mu e_n = 0$ (т.е. существует мн-во $e_n$ нулевой меры), т.ч. $f(x) \leq a + \frac{1}{n}$, $\forall x \in E \setminus e_n$

                $e = \bigcup_{n=1}^{\infty} e_n, \ \mu e = 0$ и $f(x) = a + \frac{1}{n}, \ \forall x \in E \setminus e \implies f(x) \leq a \ \forall x \in E \setminus e$.
            \end{proof}
        }
    \end{enumerate}
\end{properties}

\begin{definition}
    $L^{\infty} (E, \mu) := \{ f: E \rightarrow \bar{\mathbb{R}}$ - измеримые, т.ч. $esssup_{x\in E} |f(x)| < +\infty \}$ - векторное пр-во.

    Заведем норму для данного векторного пр-ва: $|| f ||_{\infty} := esssup_{x\in E} |f(x)|$.

    \begin{enumerate}
        \item Константа выносится
        \item Нер-во треугольника есть
        \item Функция 0 почти везде, но все же не везде
    \end{enumerate}

    % Тут лекция, видимо, закончилась
    Рассмотрим классы эквивалентности...

    \textbf{Важный частный случай ($X = \mathbb{N}$)}

    $X = \mathbb{N}$, $\mu$- считающая мера, тогда:

    1. $l^p = \{ (x_1, x_2, \ldots) \text{ - последовательность} \, : \, \sum_{k = 1}^{\infty} |x_k|^p < +\infty \}$

        Норма в данном случае: $|| x ||_p = \left( \sum_{k = 1}^{\infty} |x_k|^p \right)^{\frac{1}{p}}$

    2. $l^\infty = \{ (x_1, x_2, \ldots) \, : \, \sup |x_k| < +\infty \}$

        Норма в данном случае: $|| x ||_{\infty} = \sup_{k \in \mathbb{N}} |x_k|$
\end{definition}

\begin{theorem}
    \textbf{Вложение пространств Лебега}

    Пусть $\mu E < +\infty$ и $1 \leqslant p \leqslant q \leqslant +\infty$.

    Тогда $L^q (E, \mu) \subset L^p (E, \mu)$ и $||f||_p \leqslant ||f||_q \cdot (\mu E)^{\frac{1}{p}-\frac{1}{q}}$
\end{theorem}

\begin{proof}
    Пусть $q < +\infty$.

    Напишем неравенство Гёльдера:
    
    $\int_E |f|^p \cdot 1 \, d\mu \leqslant \left( \int_E \left(|f|^p\right)^{r} \, d\mu \right)^{\frac{1}{r}} \left( \int 1^{r'} \, d\mu \right)^{\frac{1}{r'}} = (*)$

    Здесь $\frac{1}{r} + \frac{1}{r'} = 1, \ r = \frac{q}{p} \Rightarrow \frac{1}{r'} = \frac{q - p}{q}$.

    Тогда $(*) = \left(||f||_q\right)^p \cdot (\mu E)^{\frac{q - p}{q}} \Rightarrow_{\text{извлекаем корень $p$-ой степени слева и справа}} ||f||_p \leqslant ||f||_q (\mu E)^{\frac{q - p}{pq}}$

    Пусть $q = +\infty$, тогда $||f||_p^p = \int_E |f|^p \, d\mu \leqslant \int_E ||f||_\infty^p \, d\mu = \mu E ||f||_\infty^p$

    \begin{remark}
        Для $\mu E = +\infty$ вложений нет
    \end{remark}
\end{proof}

\begin{theorem}
    $L^p (E, \mu)$ - полное, где $1 \leqslant p \leqslant +\infty$
\end{theorem}

\begin{proof}
    только для $p < +\infty$

    Пусть $f_n$ - фундаментальная последовательность функций. Мы знаем, что
    $\forall \, \varepsilon > 0 \, \exists N \, : \, \forall \, m, n \geqslant N \, : \, ||f_n - f_m||_p < \varepsilon$

    Берём $\varepsilon = \frac{1}{2}, n_1 = N$ для этого $\varepsilon$, далее берём $\varepsilon = \frac{1}{2^2}$ и $n_2 = N$ для этого $\varepsilon$ и так далее.

    Получилось, что $n_1 < n_2 < n_3 < \ldots$. А ещё $||f_{n_k} - f_n|| < \frac{1}{2^k}$ при $n \geqslant n_k$, в частности
    $||f_{n_k} - f_{n_{k + 1}}|| < \frac{1}{2^k}$ - так строили подпоследовательность.

    Тогда $\sum_{k = 1}^\infty ||f_{n_{k}} - f_{n_{k + 1}}|| < 1$.

    Заведём $S(t) = \sum_{k = 1}^\infty |f_{n_{k}}(t) - f_{n_{k + 1}}(t)|$. $S \, : \, E \to \overline{\mathbb{R}}$. Пусть
    $S_n(t)$ - частичная сумма.

    $||S_n|| \leqslant ||f_{n_1} - f_{n_2}|| + ||f_{n_2} - f_{n_3}|| + \ldots + ||f_{n_m} - f_{n_{m + 1}}|| < 1$ - норма суммы меньше суммы норм.

    $||S||^p = \int_{E} |S(t)|^p \, dt = \int_E \lim_{n \to +\infty} |S_m(t)|^p \, dt \underbrace{\leqslant}_{\text{л. Фату}} \underline{\lim} \int_E |S_m(t)|^p \, dt =
    \underline{\lim} ||S_m ||^p \leqslant 1 $

    $ \implies \int_E |S(t)|^p \, dt < +\infty \implies S(t) < +\infty$ при почти всех $t$.

    А тогда $f_{n_1} + \sum_{k = 1}^\infty (f_{n_k}(t) + f_{n_{k + 1}}(t))$ абсолютно сходится при почти всех $t \implies $ сходится при почти всех $t$.

    Пусть $S(t)$ - его сумма. Но $f_{n_k}(t)$ - частичная сумма этого ряда. Тогда
    $\lim f_{n_k} (t) = f(t)$ при почти всех $t$. Проверим, что $||f_n - f|| \rightarrow 0$.

    Возьмём $n \geqslant n_k$, тогда $||f_n - f|| \leqslant ||f_n - f_{n_k}|| + ||f_{n_k} - f|| \leqslant \frac{1}{2^k}$

    $f_{n_k} - f \rightarrow 0$ почти везде, $\lim \int_E |f_{n_k} - f|^p \, d\mu = \int E \lim |f_{n_k} - f|^p \, d\mu$, а под интегралом
    почти везде 0. Осталось понять, почему есть суммируемая мажоратна.

    $f(t) = f_{n_k} + \sum_{j = k}^\infty (f_{n_{j + 1}} - f_{n_j})$. А тогда $|f(t) - f_{n_k}| \leqslant \sum_{j = k}^\infty |f_{n_{j + 1}} - f_{n_j} | \leqslant S(t)$. Тогда $S^p$ - суммируемая мажоранта.
\end{proof}

\begin{definition}
    $(X, \rho)$ - метрическое пространство и $A \subset X$. $A$ \textbf{всюду плотно} в $X$ (или \textbf{плотно} в $X$), если $Cl A = X$.

    \begin{example}
        $X = \mathbb{R}$ и $A = \mathbb{Q}$.
    \end{example}
\end{definition}

\begin{definition}
    $f \, : \, E \to \mathbb{R}$ называется ступенчатой, если она измерима и у неё конечное число значений.
\end{definition}

\begin{lemma}
    $1 \leqslant p < +\infty, \varphi$ ступенчатая $\in L^p (E, \mu)$

    Тогда $\mu E \{ \varphi \neq 0 \} < +\infty$
\end{lemma}

\begin{proof}
    $|\varphi|$ - рассмотрим положительные значения, их конечное число, значит среди них есть наименьшее. Тогда
    на множестве $E \{ \varphi \neq 0 \}, |\varphi| \geqslant m$

    Тогда $\int_E |\varphi|^p \, d\mu = \int_{E \{ \varphi \neq 0 \}} |\varphi|^p \, d\mu \geqslant \int_{E \{ \varphi \neq 0 \}} m^p \, d\mu = m^p \mu E \{ \varphi \neq 0 \}$
\end{proof}

\begin{theorem}
    $1 \leqslant p \leqslant +\infty$

    Тогда множество ступенчатых функций из $L^p (E, \mu)$ плотно в $L^p (E, \mu)$.
\end{theorem}

\begin{proof}

    Идейно данная теорема утверждает, что любая функция из $L^p (E, \mu)$ сколь угодно хорошо может быть приближена ступенчатыми функциями.

    \begin{enumerate}
        \item {
            $p = +\infty$. Идейно: хотим д-ть, что замыкание мн-ва ступенчатых ф-й является всем мн-вом $L^p (E, \mu)$ (\textbf{def:} замыкание -- это мн-во предельных точек), т.е. нужно показать, что любая точка мн-ва $L^p (E, \mu)$ является предельной для мн-ва ступенчатых функций $\Rightarrow$ любая функция $f \in L^p (E, \mu)$ сколь угодно хорошо приближается ступенчатыми функциями.


            Возьмём $f \in L^\infty (E, \mu), f \geqslant 0$. Выберем такую $f$, что она ограничена (всегда можем так сделать, так как мы рассматривем класс эквивалентности, значит нужно выбрать ограниченного представителя данного класса).
            Тогда существует возрастающая последовательность простых $\varphi_1 \leqslant \varphi_2 \leqslant \ldots$, таких, что $\varphi_n \rightrightarrows f$ - теорема из теории меры (простые подходят под определение ступенчатых ф-й).

            Тогда $||\varphi_n - f||_{\infty} = \sup_{t \in E} |f(t) - \varphi_n(t)| \rightarrow 0$ из равномерной сходимости.

            Если $f$ проивзольная, то расписываем ее как $f = f_{+} - f_{-}$ (для них существуют $\varphi_n$ и $\psi_n$, равномерно сходящиеся к $f_{+}$ и $f_{-}$ соответственно):

            Здесь $||\varphi_n - f_{+}||_\infty \rightarrow 0$ и $||\psi_n - f_{-}||_\infty \rightarrow 0 \implies ||(f_{+} - f_{-}) - (\varphi_n - \psi_n)||_\infty \rightarrow 0$

            Таким образом, мы показали, что $\forall f \in L^{\infty} (E, \mu)$: $f$ -- предельная точка мн-ва ступенчатых функций, значит замыкание мн-ва ступенчатых ф-й действительно совпадает с $L^{\infty} (E, \mu)$.
        }
        \item {
            $p < +\infty$. Возьмём $f \in L^p (E, \mu), f \geqslant 0 \implies $ существуют ступенчатые $0 \leqslant \varphi_1 \leqslant \varphi_2 \leqslant \ldots$, такие, что $\lim \varphi_n = f$ (равномерной сходимости может не быть, так как нет условия, что $f$ ограниченная).

            $||f - \varphi_n||_p^p = \int_E |f(t) - \varphi_n(t)|^p \, d\mu \rightarrow \int_E \lim |f(t) - \varphi_n(t) |^p \, d\mu = \int_{E} 0\, d\mu = 0$.
            Опять же нужна суммируемая мажоратна (чтобы под интегралом можно было сделать предельный переход), но она есть, потому что $0 \leqslant \varphi_n \leqslant f$ ($|f(t) - \varphi_n(t)|^p$ -- неотрицательная ф-я, стремящаяся к $0$, $f^p$ -- её мажоранта). Тогда $|f - \varphi_n|^p \leqslant f^p$

            Для произвольной опять $f_{+}$ и $f_{-}$ (берем для них последовательности ф-й $\varphi_n$ и $\psi_n$):

            $|| (\varphi_n - \psi_n) - f ||_p \leq || \varphi_n - f_{+} ||_p + || \psi_n - f_{-} ||_p$
        }
    \end{enumerate}
\end{proof}

\begin{definition}
    $f: \mathbb{R}^d \rightarrow \mathbb{\bar{R}}$ - финитная функция, если она тождественно равна нулю вне некоторого компакта (т.е. мн-во $\{f \neq 0\}$ ограничено).
    \begin{example}
        Индикаторная функция отрезка
    \end{example}
\end{definition}

\begin{theorem}
    $1 \leqslant p < +\infty$ и $E \in \mathbb{R}^d$ измеримо.

    Тогда множество финитных бесконечно дифференцируемых функций плотно в $L^p (E, \lambda)$ ($\lambda$ - мера Лебега).
\end{theorem}

\begin{proof}
    Для приближения непрерывными финитными функциями.

    $f$ приближается ступенчатыми функциями, поэтому достаточно научиться приближать только их, то есть
    достаточно научится приближать функции $\mathds{1}_A$, где $A$ - измеримое и конечной меры.

    Рассмотрим $\mathds{1}_A$, найдётся $K$ - компакт и $G$ - открытое, такие, что $K \subset A \subset G$ и
    $\lambda (G \setminus K) < \varepsilon$. $\varphi (x) = \frac{d(x, \mathbb{R}^d \setminus G)}{d(x, K) + d(x, \mathbb{R}^d \setminus G)}$.

    По определению $d(x, B) = \inf_{y \in B} \rho (x, y)$.

    $\varphi(x) = 0$, если $x \not \in G$ и $\varphi(x) = 1$, если $x \in K$ и в целом $\varphi \in [0, 1]$

    Тогда $||\varphi - \mathds{1}_A||_p^p = \int_{\mathbb{R}^d} |\varphi (x) - \mathds{1}_A (x)|^p \, dx =
    \int_{G \setminus K} \underbrace{|\varphi (x) - \mathds{1}_A(x) |^p}_{\leqslant 1} \, dx \leqslant \lambda (G \setminus K) < \varepsilon$
\end{proof}

\begin{definition}
    $h \in \mathbb{R}^d, f \, : \, \mathbb{R}^d \to \overline{\mathbb{R}}$.

    Тогда $f_h$ - свдиг $f$, если $f_h(x) = f(x + h)$
\end{definition}

\begin{theorem}
    \textbf{О непрерывности сдвига}

    \begin{enumerate}
        \item {
            Если $f$ равномерно непрерывна на $\mathbb{R}^d$, то $||f_h - f||_{\infty} \rightarrow_{h \to 0} 0$
        }
        \item {
            Если $f \in L^p (\mathbb{R}^d, \lambda)$, $1 \leqslant p < +\infty$, то $||f_h - f||_p \rightarrow_{h \to 0} 0$
        }
        \item {
            Если $f \, : \, \mathbb{R} \to \mathbb{R}$ непрерывна и $2\pi$ периодична, то $||f_h - f||_{\infty} \rightarrow_{h \to 0} 0$
        }
    \end{enumerate}
\end{theorem}

\begin{proof}
    \begin{enumerate}
        \item {
            1 и 3 пункт - определение равномерной непрерывности:

            Для 1: $||f_h - f||_\infty = \sup_{x \in \mathbb{R}} |f_h(x) - f(x)| = \sup_{x \in \mathbb{R}^d} |f(x + h) - f(x)| \rightarrow 0$ - по опр. равномерной непрерывности.

            Для 3: Непрерывна + $2\pi$-периодична $\Rightarrow$ равномерно непрерывна, так как мы можем сузить рассматриваемую функцию на $2$ периода $\Rightarrow$ получим компактное мн-во. Так как непрерывная ф-я на компакте равномерно непрерывна, то получаем утв. данного пункта.
        }
        \item {
            Возьмём $\varepsilon > 0$ и $g \in C^{\infty}(\mathbb{R}^d)$ финитную функцию, что $||f - g||_p < \varepsilon$, $\{g \neq 0\} \subset B_{R}(0)$ (по предыдущей теореме мн-во финитных беск. дифф. функций плотно, значит существует $g$ сколь угодно близкая к $f$).

            $||f_h - f||_p \leqslant \underbrace{||f_h - g_h ||_p}_{\cdot \leq \varepsilon \, (*)} + ||g_h - g||_p + \underbrace{||g - f||_p}_{\cdot \leq \varepsilon} \leqslant 2\varepsilon + ||g_h - g||_p$.

            (*): Так как $f_h(x) = f(x + h)$, $g_h(x) = g(x + h)$ (аргументы сдвинули одинаково у обеих функций) и $|| f - g ||_p < \varepsilon$, то для сдвигов нер-во $\leq \varepsilon$ тоже верно.

            Теперь хотим доказать, что $||g_h - g||_p \leq \varepsilon$:

            $||g_h - g||_p^p = \int_{\mathbb{R}^d} |g_h(x) - g(x)|^p \, d\lambda = (*)$.

            $g$ нулится вне $B_R (0)$, $g_h(x) = g(x + h)$ при малых $h$ нулится вне круга $B_{R+1}(0)$, значит можно интегрировать по кругу $B_{R+1}(0)$:

            $(*) = \int_{B_{R + 1}(0)} |g(x + h) - g(x)|^p \, dx \leqslant \lambda B_{R + 1} (0) \cdot \underbrace{\sup |g(x + h) - g(x)|}_{||g_h - g||_{\infty} \rightarrow 0 \, (**)}$

            (**): Функция $g$ непрерывна и задана в круге $B_{R}(0)$, при этом данный круг является компактом $\Rightarrow$ $g$ равномерно непрерывна в данном круге. При этом вне этого круга $g$ нулится $\Rightarrow$ $g$ равномерно непрерывна в $\mathbb{R}^d$. Следовательно, можно воспользоваться 1-ым пунктом данной теоремы.
        }
    \end{enumerate}

\end{proof}
