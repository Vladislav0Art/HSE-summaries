\Subsection{Основные понятия}

\begin{definition}
    $\Omega = \{ \omega_1, \dots, \omega_n \}$ -- пространство элементарных событий (исходов).

    \begin{enumerate}
        \item равновозможные
        \item несовместные
        \item одно всегда реализуется
    \end{enumerate}
\end{definition}

\begin{definition}
    Событие $A \subset \Omega$

    $P(A) = \frac{\#A}{\#\Omega}$
\end{definition}

\begin{properties} вероятности
    \begin{enumerate}
        \item $P(\emptyset) = 0, \ P(\Omega) = 1, \ P(A) \in [0, 1]$
        \item Если $A \cap B = \emptyset$, то $P(A \cup B) = P(A) + P(B)$
        \item $\underbrace{P(A \cup B)}_{= P(A) + P(B \setminus (A \cap B))} = P(A) + P(B) - P(A \cap B)$
        \item $P(\overline{A}) = 1 - P(A)$, где $\overline{A} = \Omega \setminus A$
        \item {
            $P(A_1 \cup A_2 \dots \cup A_m) = \sum_{i = 1}^{m} P(A_i) - \sum_{i \not = j} P(A_i \cap A_j) + \sum_{i \not = j, \ i \not = k, \ j \not = k} P(A_i \cap A_j \cap A_k) - \dots + (-1)^{m-1} \cdot P(A_1 \cap \dots \cap A_m)$ -- формула включений-исключений.

            \begin{proof}
                Индукция по $m$.

                База $m = 2$.

                Переход $m \rightarrow m + 1$:

                $B_j = A_j \cup A_{m + 1}$

                $P(\underbrace{A_1 \cup \dots \cup A_m}_{=: B} \cup A_{m + 1}) = P(B \cup A_{m+1}) = \underbrace{P(B)}_{\text{это умеем расписывать по инд. предп.}} + P(A_{m+1}) - P(B \cap A_{m + 1}) = $

                $= \sum_{j = 1}^{m+1} P(A_j) - \sum_{i \not = j}^{m} P(A_i \cap A_j) + \sum_{i \not = j \not = k}^{m} P(A_i \cap A_j \cap A_k) - \underbrace{P(A_{m + 1} \cap B)}_{= P(B_1 \cup B_2 \dots \cup B_m)}$, где $B_i := A_i \cap A_{m + 1}$.
            \end{proof}
        }
        \item {
            $P(A \cup B) \leq P(A) + P(B)$

            $P(A_1 \cup \dots \cup A_m) \leq \sum_{j = 1}^{m} P(A_j)$
        }
    \end{enumerate}
\end{properties}


\begin{definition}
    \textbf{Условная вероятность}.

    $B \not = \emptyset, \ P(B) > 0$.

    Знаем, что выполнилось событие $B$, хотим узнать вероятность наступления $A$.

    $P(A | B) = \frac{\#(A \cap B)}{\#B} = \frac{\frac{\# (A \cap B)}{\#\Omega}}{\frac{\#B}{\#\Omega}} = \frac{P(A \cap B)}{P(B)}$
\end{definition}

\begin{properties}
    \begin{enumerate}
        \item $P(A | A) = 1$, если $B \subset A$, то $P(A | B) = 1$
        \item {
            Если $A_1 \cap A_2 = \emptyset$, то $P(A_1 \cup A_2 | B) = P(A_1 | B) + P(A_2 | B)$

            В частности: $P(A | B) + P(\overline{A} | B) = 1$
        }
    \end{enumerate}
\end{properties}
\begin{remark}
    $P(A | B) + P(A | \overline{B})$ не обязана быть 1.

    Пример: игральный кубик, $B$ -- выпало четное число, $A$ -- выпало кратное трем.

    $P(A | B) = \frac{1}{3}, \ P(A | \overline{B}) = \frac{1}{3}$
\end{remark}

\begin{theorem}
    \textbf{Формула полной вероятности}.

    Пусть $\Omega = \bigsqcup_{j=1}^{m} B_j, \ P(B_j) > 0$.

    Тогда $P(A) = \sum_{j=1}^{m} P(A | B_j) \cdot P(B_j)$
\end{theorem}

\begin{proof}
    $\sum_{j=1}^{m} \underbrace{P(A | B_j)}_{= \frac{P(A \cap B_j)}{P(B_j)}} \cdot P(B_j) = \sum_{j=1}^{m} P(A \cap B_j) = P(A \cap \bigsqcup_{j=1}^{m} B_j) = P(A)$
\end{proof}

\begin{example}

    Пусть есть 2 урны с шариками:

    \begin{enumerate}
        \item[I.] $3$ белых шара, $5$ черных шаров
        \item[II.] $5$ белых, $5$ черных
    \end{enumerate}

    $2$ шара из $I$ положили в $II$, затем вынули 1 шар из $II$, $P(\text{вынули белый}) = ?$

    $A$ -- вынули из $II$ белый шар.

    $B_0, \ B_1, \ B_2$, где $B_j$ -- переложили $j$ белых шаров из $I$ в $II$.

    Тогда $P(A | B_0) = \frac{5}{12}, \ P(A | B_1) = \frac{1}{2}, \ P(A | B_2) = \frac{7}{12}$.

    $P(B_0) = \frac{C_5^2}{C_8^2} = \frac{5}{14}$

    $P(B_1) = \frac{15}{C_8^2} = \frac{15}{28}$

    $P(B_2) = \frac{C_3^2}{C_8^2} = \frac{3}{28}$

    Подставляем в формулу:

    $P(A) = \frac{5}{12} \cdot \frac{5}{14} + \frac{1}{2}\cdot \frac{15}{28} + \frac{7}{12}\cdot \frac{3}{28} = \frac{23}{48}$
\end{example}

\begin{theorem}
    \textbf{Формула Байеса}.

    Пусть $P(A) > 0, \ P(B) > 0$, тогда $P(B | A) = \frac{P(A | B) \cdot P(B)}{P(A)}$
\end{theorem}
\begin{proof}
    Расписываем $P(A | B)$, получаем в правой части: $\frac{P(A \cap B)}{P(B)} \cdot P(B) \cdot \frac{1}{P(A)}$.
\end{proof}

\begin{theorem}
    \textbf{Байеса}.

    Пусть $P(A) > 0, \ P(B_j) > 0, \ \Omega = \bigsqcup_{j=1}^{m} B_j$, тогда

    $P(B_j | A) = \frac{P(A | B_j) \cdot P(B_j)}{P(A|B_1) P(B_1) + \dots + P(A|B_m) P(B_m)}$
\end{theorem}

\begin{example}
    Есть $2$ монеты (одна симметричная, вторая $P(\text{орла}) = \frac{1}{3}, \ P(\text{решка}) = \frac{2}{3}$). Взялу наугад монету, побросили и выпал орел. Какова вероятность, что мы взяли симметричную монету?

    $A$ -- выпал орел, $B$ -- монета симметричная ($\overline{B}$ -- монета кривая).

    $P(B | A) = \frac{P(A | B) P(B)}{P(A|B) P(B) + P(A | \overline{B}) P(\overline{B})} = \frac{\frac{1}{2} \cdot \frac{1}{2}}{\frac{1}{2} \cdot \frac{1}{2} + \frac{1}{3} \cdot \frac{1}{2}} = \frac{3}{5}$
\end{example}



\begin{definition}
    \textbf{Независимые события}.

    \textbf{Рассуждения:} $A$ не зависит от $B$, если $P(A) = P(A | B) = \frac{P(A \cap B)}{P(B)}$.

    \textbf{Def:} $A, \ B$ независимые события, если $P(A \cap B) = P(A) \cdot P(B)$
\end{definition}

\begin{definition}
    События $A_1, A_2, \dots, A_m$ -- независимы в совокупности, если

    $P(A_{i_1} \cap A_{i_2} \cap \dots \cap A_{i_k}) = P(A_{i_1}) \cdot P(A_{i_2}) \cdot \dots \cdot P(A_{i_k})$ -- для любых индексов $i_j$.
\end{definition}

\begin{remark}
    Независимость в совокупности $\implies$ попарная независимость.

    Наоборот неверно.
\end{remark}

\begin{example}
    Есть два игральных кубика.

    $A$ -- на первом кубике выпало четное число.

    $B$ -- на втором выпало четное число.

    $C$ -- сумма на кубиках четная.

    Пространство элементарных исходов это все пары $(i, j)$, где $i, j \in \{ 1, 2, 3, 4, 5, 6 \}$, $\#\Omega = 36$.

    $P(A) = \frac{1}{2}, \ P(B) = \frac{1}{2}, \ P(C) = \frac{1}{2}$.

    $A \cap B = A \cap C = B \cap C = A \cap B \cap C$.

    $P(A \cap B) = \frac{1}{4} = \frac{1}{2} \cdot \frac{1}{2} = P(A) \cdot P(B)$, остальные равенства тоже выполняются $\implies$ попарная независимость.

    $P(A \cap B \cap C) = \frac{1}{4} \not = \frac{1}{2} \cdot \frac{1}{2} \cdot \frac{1}{2} = P(A) \cdot P(B) \cdot P(C) \implies$ нет независимости в совокупности.
\end{example}

\begin{exerc}
    Д-ть, что $A_1, \dots A_m$ независимы в совокупности $\Leftrightarrow P(B_1 \cap B_2 \cap \dots \cap B_m) = P(B_1) \dots P(B_m)$, где $B_j = A_j$ или $\overline{A_j}$ (все $2^m$ равенств).
\end{exerc}

\begin{remark}
    Небольшое обобщение.

    $\Omega = \{ \omega_1, \dots, \omega_n \}$ -- пр-во элементарных исходов.

    Также у нас есть $p_1, \dots p_n: \ \sum_{i=1}^{n} p_i = 1, \ \forall i: \ p_i \geq 0$.

    $P(A) = \sum_{j: \ \omega_j \in A} p_j$.
\end{remark}

\begin{theorem}
    \textbf{Схема Бернулли}.

    орел $=$ успех $=$ 1.

    решка $=$ неудача $=$ 0.

    $P(\text{орел}) = p, \ 0 \leq p \leq 1$

    $P(\text{решка}) = 1 - p$

    Бросаем монету $n$ раз, получаем последовательность исходов:

    $\Omega = \{ x_1, x_2, \dots, x_n \}: \ x_j = 0 \text{ или } 1$.

    $\omega = (x_1, x_2, \dots, x_n), \ P(\{ \omega \}) = p^{\# i: \ x_i = 1} \cdot q^{\# i: \ x_i = 0} = p^{\sum x_i} \cdot q^{n - \sum x_i}$

    Хотим узнать:

    $P(\text{выпало ровно } k \text{ орлов}) = C_{n}^{k} p^k q^{n - k}$

    $P(i\text{-ое подбрасывание }) = P(\text{орел}) = p$ -- независимые в совокупности по $i = 1, 2, \dots, n$.
\end{theorem}

\begin{theorem}
    \textbf{Полиномиальная схема}.

    $p_1, p_2, \dots, p_m: \sum p_i = 1$.

    $P(x_i = k) = p_k$, где $x_i \in \{ 1, 2, \dots, m \}$

    $\Omega = \{(x_1, x_2, \dots, x_n)\}, \ \omega = (x_1, x_2, \dots, x_n)$

    $P(\{ \omega \}) = p_1^{\# \{ i: x_i = 1 \}} \cdot \dots \cdot p_m^{\# \{ i: x_i = m \}}$

    $k_1 + k_2 + \dots + k_m = n$


    $P(k_1 \text{ раз выпало 1, } k_2 \text{ раз выпало 2, ...}) = \underbrace{\binom{n}{k_1, k_2, \dots, k_m}}_{= \frac{n!}{k_1! \dots k_m!}} \cdot p_1^{k_1} \cdot \dots \cdot p_m^{k_m}$
\end{theorem}

\begin{theorem}
    \textbf{Эрдёша-Мозера}

    Рассмотрим турнир на $n$ команд. При каком наибольшем $k$ можно
    всегда выбрать команды $A_1, A_2 \ldots A_k$, так, что $A_i$
    выиграла у $A_j$, если $i < j$? При $k \leqslant 1 +  [2 \log_2 n] $
\end{theorem}

\begin{proof}
    Предположим, что $k \geqslant 2 + [2 \log_2 n] > 1 + 2 \log_2 n$. Хотим показать, что
    при таких $k$ точно найдётся турнир, в котором нельзя выбрать $k$ команд.

    Рассмотрим случайный турнир(Всего встреч $\binom{n}{2}$, тогда
    $2^{\binom{n}{2}}$ разных турниров. Случайный - берём из этой
    кучи наугад).

    $P(\text{A выиграла у B}) = \frac{1}{2}$.

    Рассмотрим $A_1, A_2, \ldots A_k$ команды.

    \begin{enumerate}
        \item {
            $P(A_1, A_2 \ldots A_k  \text{ подходят}) = (\frac{1}{2})^{\binom{k}{2}}$.
        }
        \item {
            $P(A_1, A_2 \ldots A_k \text{ можно переименовать, так, что они подошли}) \leqslant \frac{k!}{2^{\binom{k}{2}}}$
        }
        \item {
            $P(\text{какие-то $k$ команд подошли}) \leqslant \binom{n}{k} \cdot \frac{k!}{2^{\binom{k}{2}}}$
        }
    \end{enumerate}

    Нужно понять, что если $k \geqslant 2 + [2\log_2 n]$, то $\binom{n}{k} \frac{k!}{2^{\binom{k}{2}}} < 1$.

    Действительно, $\binom{n}{k} \frac{k!}{2^{\binom{k}{2}}} = \frac{n(n - 1)(n - 2) \ldots (n - k + 1)}{2^{\frac{k(k - 1)}{2}}} < \frac{n^k}{(2^{\frac{k - 1}{2}})^k} =
    \left( \frac{n}{2^{\frac{k - 1}{2}}}\right)^k$

    Мы знаем, что $k > 1 + 2 \log_2 n \Leftrightarrow \frac{k - 1}{2} > \log_2 n \implies 2^{\frac{k - 1}{2}} > n$. И тогда $\left( \frac{n}{2^{\frac{k - 1}{2}}}\right)^k < 1$.
    Это значит, что вероятность, что никакие команды не подходят - положительная, значит есть турнир, в котором $k$ команд выбрать нельзя.
\end{proof}

\Subsection{Предельные теоремы для схем Бернулли}

\begin{definition}
    Схема Бернулли с вероятностью успеха $p \in (0, 1)$. $S_n$ - число успехов при $n$ испытаниях. $P(S_n = k) = \binom{n}{k} p^k q^{n - k}$

    Что будет больше $P(S_{1000} = 220)$ при $p = \frac{1}{5}$ или $P(S_{2000}) = 360$ при $p = \frac{1}{6}$. Точные вычисления дают $0.008984$ и
    $0.006625$ соответственно.

\end{definition}

\begin{theorem}
    \textbf{Пуассона}

    Схема Бернулли с $n$ испытаниями и вероятностью успеха $p_n$ - зависит от $n$.
    Если $np_n \rightarrow \lambda > 0$. Тогда $P(S_n = k) \rightarrow \frac{\lambda^k}{k!} e^{-\lambda}$
\end{theorem}

\begin{remark}
    Если $n p_n = \lambda$, то теорема верна при $k = o(\sqrt{n})$
\end{remark}

\begin{proof}
    $P(S_n = k) = \binom{n}{k} p^k (1 - p)^{n - k} = \frac{n(n - 1)\ldots(n - k + 1)}{k!} p^k (1 - p)^{n - k}$
    $\sim \frac{n^k}{k!} p^k (1 - p)^{n - k} = \frac{(np)^k}{k!} (1 - p)^{n - k} \sim \frac{\lambda^k}{k!} (1 - p)^{n - k}$.

    Осталось показать, что $(1 - p)^{n - k} \sim e^{-\lambda}$. Прологарифмируем: $\ln (1 - p)^{n - k} = \\ = (n - k) \ln (1 - p)$
    $\sim -n p \sim -\lambda$

    Доказательство замечания:

    Нам нужно показать, что $n(n - 1)\ldots(n - k + 1) \sim n^k$, все остальные переходы будут верны.

    $\frac{n(n - 1) \ldots (n - k + 1)}{n^k} = 1 \cdot (1 - \frac{1}{n}) \ldots \cdot (1 - \frac{k - 1}{n}) \underbrace{\geqslant}_{(*)}$
    $1 - \frac{1}{n} - \ldots -  \frac{k - 1}{n} = 1 - \frac{k(k - 1)}{2n} \rightarrow 1$

    $(*)$ Неравенство $(1 - x_1) \ldots (1 - x_k) \geqslant 1 - x_1 - x_2 - \ldots - x_k$ при $0 \leqslant x_i \leqslant 1$ - индукция.
\end{proof}

\begin{theorem}
    \textbf{Прохорова}

    Если $\lambda = np$, то $\sum_{k = 0}^{+\infty} |P(S_n = k) - \frac{\lambda^k}{k!}e^{-\lambda}| \leqslant \frac{2\lambda}{n} \cdot \min(2, \lambda)$

\end{theorem}

\begin{example}
    Игра в рулетку: 36 чисел и ноль.

    $p = \frac{1}{37}, n = 111, np = 3 = \lambda$.

    $P(S_{111} = 3) = \binom{111}{3} (\frac{1}{37})^3 (1 - \frac{1}{37})^{111 - 3} = 0.227127$

    Из Пуассона $\frac{\lambda^3}{3!} e^{-\lambda} = 0.224$

    Видим, что приближение хорошее.

    $P(\text{ставка удачная хотя бы 4 раза}) = 1 - P(S_{111} = 0) - P(S_{111} = 1) - P(S_{111} = 2) - P(S_{111} = 3) = $
    $1 - \frac{\lambda^0}{0!}e^{-\lambda} -$
    $\frac{\lambda^1}{1!}e^{-\lambda} - \frac{\lambda^2}{2!}e^{-\lambda} - \frac{\lambda^3}{3!}e^{-\lambda} = 0.352754$

    А по формулам $0.352768$
\end{example}

\begin{theorem}
    \textbf{Локальная предельная теорема Муавра-Лапласа}

    Схема Бернулии с вероятностью успеха $p \in (0, 1), \ q = 1 - p, \ x = \frac{k - np}{\sqrt{npq}}$, где $k$ зависит от $n$, и $n$ меняется. $|x| \leqslant T$ -- при $n \rightarrow +\infty$ и любых $k$. Тогда:

    $P(S_n = k) \sim_{n \to +\infty} \frac{1}{\sqrt{2\pi n p q}} e^{\frac{-x^2}{2}}$

\end{theorem}

\begin{proof}
    \hfill \smallbreak
    \begin{enumerate}
        \item {
            $k = np + x\sqrt{npq} \geqslant np - T\sqrt{npq} \rightarrow +\infty$
        }
        \item {
            $n - k = nq - x\sqrt{npq} \geqslant nq - T\sqrt{npq} \rightarrow +\infty$
        }
    \end{enumerate}

    $P(S_n = k) = \binom{n}{k} p^k q^{n - k} = \frac{n!}{k!(n - k)!}p^k q^{n - k}$

    Напишем формулу Стирлинга ($n! \sim_{n \to \infty} \sqrt{2 \pi n} \cdot n^n e^{-n}$):

    $\frac{n^n e^{-n} \sqrt{2\pi n}p^kq^{n-k}}{k^k e^{-k} \sqrt{2\pi k} \cdot (n - k)^{n - k} e^{-(n - k)} \sqrt{2 \pi (n - k)}} =$
    $\frac{p^k q^{n - k}}{(\frac{k}{n})^k \cdot (\frac{n - k}{n})^{n - k} \cdot \sqrt{2 \pi \frac{k}{n} (1 - \frac{k}{n})n}}$.


    Заметим, что $\frac{k}{n} = p + \frac{x\sqrt{pq}}{\sqrt{n}} \rightarrow p$ \\ и $\frac{n - k}{n} \rightarrow q$

    Поэтому остаётся доказать, что $\frac{(\frac{k}{n})^k \cdot (\frac{n - k}{n})^{n - k}}{p^k q^{n - k}} \rightarrow e^{\frac{x^2}{2}}$. Прологарифмируем:

    Получим: $k \ln \frac{k}{n} + (n - k) \ln \frac{n - k}{n} - k \ln p - (n - k)\ln q \rightarrow \frac{x^2}{2}$

    Введём обозначения: $\alpha = \frac{k}{n} \rightarrow p, \beta = \frac{n - k}{n} \rightarrow q$. Тогда $k = n\alpha, n - k = n\beta$ и всё перепишется в виде:

    $n \alpha \ln \alpha + n \beta \ln \beta - n\alpha \ln p - n\beta \ln q = \underbrace{n\alpha \ln \frac{\alpha}{p} + n \beta \ln \frac{\beta}{q}}_{(*)} \rightarrow \frac{x^2}{2}$

    Мы знаем, что $\frac{\alpha}{p} = 1 + x\sqrt{\frac{q}{np}}$ и $\frac{\beta}{q} = 1 - x\sqrt{\frac{p}{nq}}$ - из первых двух тождеств в доказательстве.

    Напишем Тейлора:

    $ln(1+t) = t - \frac{t^2}{2} + o(t^2)$

    $\ln \frac{\alpha}{p} = \ln (1 + x\sqrt{\frac{q}{np}}) = x\sqrt{\frac{q}{np}} - \frac{1}{2}x^2\frac{q}{np} + o(\frac{1}{n}) $

    $\ln \frac{\beta}{q} = \ln (1 - x\sqrt{\frac{p}{nq}}) = -x\sqrt{\frac{p}{nq}} - \frac{1}{2}x^2 \frac{p}{nq} + o(\frac{1}{n})$

    Тогда $(*) = x\sqrt{pq}\sqrt{n} + x^2q - \frac{1}{2}x^2q + o(\frac{1}{n}) - x\sqrt{pq}\sqrt{n} + x^2p - \frac{1}{2}x^2p + o(\frac{1}{n}) = $
    $x^2 (\frac{q}{2} + \frac{p}{2}) + o(1) = \frac{x^2}{2} + o(1)$
\end{proof}

\begin{remark}
    Если $\varphi (n) = o(n^\frac{2}{3})$ и $|k - np| \leqslant \varphi (n)$, то теорема тоже верна
\end{remark}

\begin{example}
    Всё та же рулетка. $n = 222, k = 111$. Пытаемся ставить на четное/нечётное(кроме 0). $p = \frac{18}{37}$

    $P(S_{222} = 111) \approx \frac{1}{\sqrt{2 \pi n p q}} e^{\frac{-x^2}{2}} \approx 0.049395\ldots$

    Если считать точно, то получим $0.0493228\ldots$
\end{example}

\begin{theorem}
    \textbf{Интегральная теорема Муавра-Лапласа}

    $0 < p < 1$. $P(a < \frac{S_n - np}{\sqrt{npq}} \leqslant b) \rightarrow_{n \to \infty} \frac{1}{\sqrt{2\pi}} \int_a^b e^{\frac{-t^2}{2}} \, dt$

    Стремление равномерно по $a, \ b \in \mathbb{R}$.
\end{theorem}

\begin{theorem}
    \textbf{Берри-Эссеена}

    Обозначение: $\Phi (x) = \frac{1}{\sqrt{2\pi}} \int_{-\infty}^x e^{-\frac{t^2}{2}} \, dt$, $\Phi_0 (x) = \frac{1}{\sqrt{2\pi}} \int_{0}^x e^{-\frac{t^2}{2}} \, dt$

    $\sup_{x\in \mathbb{R}}\left| P(\frac{S_n - np}{\sqrt{npq}} \leqslant x) - \Phi (x) \right| \leqslant \frac{p^2 + q^2}{\sqrt{npq}} \cdot \frac{1}{2}$

    \begin{remark}
        Константа лучше, чем $\frac{c}{\sqrt{n}}$ не бывает.
    \end{remark}

    \begin{remark}
        $P(a < S_n \leqslant b) = P(\frac{a - np}{\sqrt{npq}} < \frac{S_n - np}{\sqrt{npq}} \leqslant \frac{b - np}{\sqrt{npq}}) \to
        \Phi (\frac{b - np}{\sqrt{npq}}) - \Phi (\frac{a - np}{\sqrt{npq}})$

        Отсюда получили, что лучше всего писать полуцелые $a$ и $b$.
    \end{remark}

    \begin{remark}
        Если $p$ или $q$ очень маленькие, то произведение $np$ маленькое и оценка будет плохой. В таких случаях хорошо
        использовать Пуассона. Муавра-Лаплас же хорош, когда $np$ большое.
    \end{remark}
\end{theorem}

\begin{example}
    $p = q = \frac{1}{2}$. Вопрос: $P(S_{2n} = n) = \binom{2n}{n} \frac{1}{2^{2n}} \sim
    \frac{4^n}{\sqrt{\pi n}} \frac{1}{4^n} = \frac{1}{\sqrt{\pi n}}$.

    Но $P(S_{2n} < n) = P(S_{2n} > n)$.

    Тогда $P(S_{2n} \leqslant n) = \frac{1 + P(S_{2n} = n)}{2} = \frac{1}{2} + \frac{1}{2\sqrt{\pi n}} + o(\frac{1}{\sqrt{n}})$

    Муавра-Лаплас нам говорит, что $P(S_{2n} \leqslant n) \to \frac{1}{\sqrt{2\pi}} \int_{-\infty}^0 e^{-\frac{t^2}{2}} \, dt = \frac{1}{2}$

    Но $P(S_{2n} \leqslant n) = \frac{1}{2} + \frac{1}{2\sqrt{\pi n}} + o(\frac{1}{\sqrt{n}})$
\end{example}

\begin{example}
    \textbf{Задача о театре}

    Есть театр и 2 входа. У каждого входа расположен гардероб.
    В театре $n = 1600$ мест. Хотим сделать размер гардероба как можно меньше, но чтобы
    переполнения случались как можно реже (не чаще, чем раз в месяц).

    Пусть $c$ мест в итоге в гардеробе.

    За успех считаем ситуацию, когда человек вошел в театр и пошел в ближайший к нему гардероб (т.е. в ближайшем гардеробе было место, и человек не пошел в дальний гардероб). Пусть $S_n$ -- кол-во успешных испытаний.

    Так как в каждый гардероб мы допускаем $c$ мест, то кол-во успехов $S_n \leq c$.

    $p = q = \frac{1}{2}$. Нужно, чтобы $n - c \leqslant S_n \leqslant c$. И $P(n - c \leqslant S_n \leqslant c) > \frac{29}{30}$ -- т.е. хотя бы в 29 днях из 30 ближайший к каждому входу гардероб не переполняется.

    $P(n - c \leqslant S_n \leqslant c) = P\left(\frac{n - c - \frac{n}{2}}{\sqrt{n \cdot \frac{1}{4}}}
    \leqslant \frac{S_n - \frac{n}{2}}{\sqrt{n \cdot \frac{1}{4}}} \leqslant \frac{c - \frac{n}{2}}{\sqrt{n \cdot \frac{1}{4}}}\right) =$


    $P\left(\frac{800 - c}{20} \leqslant \frac{S_n - 800}{20} \leqslant \frac{c - 800}{20}\right) \to \Phi (\frac{800 - c}{20}) - \Phi (\frac{c - 800}{20}) =
    \frac{1}{\sqrt{2\pi}} \int_{\frac{800 - c}{20}}^{\frac{c - 800}{20}} e^{-\frac{t^2}{2}} \, dt = 2 \cdot \Phi_0(\frac{c - 800}{20}) > \frac{29}{30}$

    $\Phi_0 (\frac{c - 800}{20}) > \frac{29}{60} \implies c = 843$.

\end{example}

\begin{example}
    \textbf{Случайное блуждание на прямой}

    Есть прямая, будем считать, что у нас блуждания исключительно по целым точкам.

    В каждой точке подбрасываем монетку. С вероятностью $p$ идём вперёд, $q$ - идём назад.

    $a_{n + 1} = a_{n} + 1$ с вероятностью $p$

    $a_{n + 1} = a_{n} - 1$ с вероятностью $q$

    $a_n \equiv n \mod 2$

    Если представить, что шаг влево -- это $0$, а шаг в право -- это $1$, то сдвиг из точки $a_n$ будет определяться как $2 \cdot x - 1$, где $x = 0, \ 1$ в зависимости от того, в какую сторону идем (т.е. $a_{n+1} = a_n + (2\cdot x - 1)$).

    Тогда пусть $S_n$ -- кол-во единичек, тогда $a_n = 2\cdot S_n - n$.

    Пусть мы всегда стартуем с $a_0 = 0$, тогда определим, чему равна вероятность попасть за $n$ шагов в точку $k$ ($a_n = k$):

    $P(a_n = k) = P(S_n = \frac{n + k}{2}) = \begin{cases}
        0 \text{, если } n \not \equiv k \mod 2 \\
        \binom{n}{\frac{n + k}{2}} p^{\frac{n + k}{2}} q^{\frac{n - k}{2}} \text{, иначе}
    \end{cases}$
\end{example}

\begin{theorem}
    \textbf{ван дер Вардена}

    Рассмотрим числа $1, 2 \ldots k$ и покрасим их в 2 цвета.

    Тогда существует $k_n$, такое, что, если $k > k_n$, то при любой раскраске
    найдётся одноцветная $n$-членная арифметическая прогрессия.
\end{theorem}

\begin{theorem}
    \textbf{Эрдеша-Радо}

    $k_{n + 1} \geqslant \sqrt{n \cdot 2^{n + 1}}$
\end{theorem}

\begin{proof}
    $A_1, A_2 \ldots A_m$ - все арифметические прогрессии длины $n + 1$ из чисел
    $1, 2 \ldots k$.

    С разностью $1 \, : \, k - n$ прогрессий.

    С разностью $2 \, : \, k - 2n$ прогрессий.

    $\ldots$

    С разностью $[\frac{k}{n}] \, : \, k - [\frac{k}{n}] \cdot n$ прогрессий

    Тогда $m = (k - n) + (k - 2n) + \ldots + k - [\frac{k}{n}] \cdot n = k \cdot [\frac{k}{n}] - n \cdot \frac{[\frac{k}{n}] \cdot ([\frac{k}{n}] + 1)}{2} =
    [\frac{k}{n}] (k - \frac{1}{2}n ([\frac{k}{n}] + 1)) < \frac{k}{n} (k - \frac{1}{2} \cdot n \cdot \frac{k}{n}) = \frac{k^2}{2n}$ - это
    оценка сверху.

    $P(\text{$A_i$ - одноцветная}) = 2 \cdot \frac{1}{2^{n + 1}} = \frac{1}{2^n}$ (2 - выбор цвета).

    $P(\text{какое-то $A_i$ - одноцветно}) = \sum_{i = 1}^m P(\text{$A_i$ - одноцветно}) = \frac{m}{2^n} <
    \frac{k^2}{2n} \cdot \frac{1}{2^n} = (\frac{k}{\sqrt{2^{n + 1} \cdot n}})^2 \leqslant 1$
    (если так, то найдётся, на которой не выполнится)
\end{proof}

%---------------------%
% Все выше поправлено %
%---------------------%