\Subsection{Основные понятия}

\begin{definition}
    $\Omega = \{ \omega_1, \dots, \omega_n \}$ -- пространство элементарных событий (исходов).

    \begin{enumerate}
        \item равновозможные
        \item несовместные
        \item одно всегда реализуется
    \end{enumerate}
\end{definition}

\begin{definition}
    Событие $A \subset \Omega$

    $P(A) = \frac{\#A}{\#\Omega}$
\end{definition}

\begin{properties} вероятности
    \begin{enumerate}
        \item $P(\emptyset) = 0, \ P(\Omega) = 1, \ P(A) \in [0, 1]$
        \item Если $A \cap B = \emptyset$, то $P(A \cup B) = P(A) + P(B)$
        \item $\underbrace{P(A \cup B)}_{= P(A) + P(B \setminus (A \cap B))} = P(A) + P(B) - P(A \cap B)$
        \item $P(\overline{A}) = 1 - P(A)$, где $\overline{A} = \Omega \setminus A$
        \item {
            $P(A_1 \cup A_2 \dots \cup A_m) = \sum_{i = 1}^{m} P(A_i) - \sum_{i \not = j} P(A_i \cap A_j) + \sum_{i \not = j, \ i \not = k, \ j \not = k} P(A_i \cap A_j \cap A_k) - \dots + (-1)^{m-1} \cdot P(A_1 \cap \dots \cap A_m)$ -- формула включений-исключений.

            \begin{proof}
                Индукция по $m$.
                
                База $m = 2$.

                Переход $m \rightarrow m + 1$:

                $B_j = A_j \cup A_{m + 1}$

                $P(\underbrace{A_1 \cup \dots \cup A_m}_{=: B} \cup A_{m + 1}) = P(B \cup A_{m+1}) = \underbrace{P(B)}_{\text{это умеем расписывать по инд. предп.}} + P(A_{m+1}) - P(B \cap A_{m + 1}) = $

                $= \sum_{j = 1}^{m+1} P(A_j) - sum_{i \not = j}^{m} P(A_i \cap A_j) + \sum_{i \not = j \not = k}^{m} P(A_i \cap A_j \cap A_k) - \underbrace{P(A_{m + 1} \cap B)}_{= P(B_1 \cup B_2 \dots \cup B_m)}$, где $B_i := A_i \cap A_{m + 1}$. 
            \end{proof}
        }
        \item {
            $P(A \cup B) \leq P(A) + P(B)$

            $P(A_1 \cup \dots \cup A_m) \leq \sum_{j = 1}^{m} P(A_j)$
        }
    \end{enumerate}
\end{properties}


\begin{definition}
    \textbf{Условная вероятность}.

    $B \not = \emptyset, \ P(B) > 0$.

    Знаем, что выполнилось событие $B$, хотим узнать вероятность наступления $A$.

    $P(A | B) = \frac{\#(A \cap B)}{\#B} = \frac{\frac{\# (A \cap B)}{\#\Omega}}{\frac{\#B}{\#\Omega}} = \frac{P(A \cap B)}{P(B)}$
\end{definition}

\begin{properties}
    \begin{enumerate}
        \item $P(A | A) = 1$, если $B \subset A$, то $P(A | B) = 1$
        \item {
            Если $A_1 \cap A_2 = \emptyset$, то $P(A_1 \cup A_2 | B) = P(A_1 | B) + P(A_2 | B)$

            В частности: $P(A | B) + P(\overline{A} | B) = 1$
        }
    \end{enumerate}
\end{properties}
\begin{remark}
    $P(A | B) + P(A | \overline{B})$ не обязана быть 1.

    Пример: игральный кубик, $B$ -- выпало четное число, $A$ -- выпало кратное трем.

    $P(A | B) = \frac{1}{3}, \ P(A | \overline{B}) = \frac{1}{3}$
\end{remark}

\begin{theorem}
    \textbf{Формула полной вероятности}.

    Пусть $\Omega = \bigsqcup_{j=1}^{m} B_j, \ P(B_j) > 0$.

    Тогда $P(A) = \sum_{j=1}^{m} P(A | B_j) \cdot P(B_j)$
\end{theorem}

\begin{proof}
    $\sum_{j=1}^{m} \underbrace{P(A | B_j)}_{= \frac{P(A \cap B_j)}{P(B_j)}} \cdot P(B_j) = \sum_{j=1}^{m} P(A \cap B_j) = P(A \cap \bigsqcup_{j=1}^{m} B_j) = P(A)$
\end{proof}

\begin{example}
    \begin{enumerate}
        \item[I.] $3$ белых шара, $5$ черных шаров
        \item[II.] $5$ белых, $5$ черных  
    \end{enumerate}

    $2$ шара из $I$ положили в $II$, затем вынули 1 шар из $II$, $P(\text{вынули белый}) = ?$

    $A$ -- вынули из $II$ белый шар.

    $B_0, \ B_1, \ B_2$, где $B_j$ -- переложили $j$ белых шаров из $I$ в $II$.

    Тогда $P(A | B_0) = \frac{5}{12}, \ P(A | B_1) = \frac{1}{2}, \ P(A | B_2) = \frac{7}{12}$.
    
    $P(B_0) = \frac{C_5^2}{C_8^2} = \frac{5}{14}$

    $P(B_1) = \frac{15}{C_8^2} = \frac{15}{28}$
    
    $P(B_2) = \frac{C_3^2}{C_8^2} = \frac{3}{28}$

    Подставляем в формулу:

    $P(A) = \frac{331}{336}$
\end{example}

\begin{theorem}
    \textbf{Формула Байеса}.

    Пусть $P(A) > 0, \ P(B) > 0$, тогда $P(B | A) = \frac{P(A | B) \cdot P(B)}{P(A)}$
\end{theorem}
\begin{proof}
    Расписываем $P(A | B)$, получаем в правой части: $\frac{P(A \cap B)}{P(B)} \cdot P(B) \cdot \frac{1}{P(A)}$.
\end{proof}

\begin{theorem}
    \textbf{Байеса}.

    Пусть $P(A) > 0, \ P(B_j) > 0, \ \Omega = \bigsqcup_{j=1}^{m} B_j$, тогда 

    $P(B_j | A) = \frac{P(A | B_j) \cdot P(B_j)}{P(A|B_1) P(B_1) + \dots + P(A|B_m) P(B_m)}$
\end{theorem}

\begin{example}
    Есть $2$ монеты (одна симметричная, вторая $P(\text{орла}) = \frac{1}{3}, \ P(\text{решка}) = \frac{2}{3}$). Взялу наугад монету, побросили и выпал орел. Какова вероятность, что мы взяли симметричную монету?

    $A$ -- выпал орел, $B$ -- монета симметричная ($\overline{B}$ -- монета кривая).

    $P(B | A) = \frac{P(A | B) P(B)}{P(A|B) P(B) + P(A | \overline{B}) P(\overline{B})} = \frac{\frac{1}{2} \cdot \frac{1}{2}}{\frac{1}{2} \cdot \frac{1}{2} + \frac{1}{3} \cdot \frac{1}{2}} = \frac{3}{5}$
\end{example}



\begin{definition}
    \textbf{Независимые события}.

    Рассуждения: $A$ не зависит от $B$, если $P(A) = P(A | B) = \frac{P(A \cap B)}{P(B)}$.

    Опр. $A, \ B$ независимые события, если $P(A \cap B) = P(A) \cdot P(B)$
\end{definition}

\begin{definition}
    События $A_1, A_2, \dots, A_m$ -- независимы в совокупности, если 

    $P(A_{i_1} \cap A_{i_2} \cap \dots \cap A_{i_k}) = P(A_{i_1}) \cdot P(A_{i_2}) \cdot \dots \cdot P(A_{i_k})$ -- для любых индексов $i_j$.
\end{definition}

\begin{remark}
    Независимость в совокупности $\implies$ попарная независимость.

    Наоборот неверно.
\end{remark}

\begin{example}
    Есть два игральных кубика.

    $A$ -- на первом кубике выпало четное число.

    $B$ -- на втором выпало четное число.

    $C$ -- сумма на кубиках четная.

    Пространство элементарных исходов это все пары $(i, j)$, где $i, j \in \{ 1, 2, 3, 4, 5, 6 \}$, $\#\Omega = 36$.

    $P(A) = \frac{1}{2}, \ P(B) = \frac{1}{2}, \ P(C) = \frac{1}{2}$.

    $A \cap B = A \cap C = B \cap C = A \cap B \cap C$.

    $P(A \cap B) = \frac{1}{4} = \frac{1}{2} \cdot \frac{1}{2} = P(A) \cdot P(B)$, остальные равенства тоже выполняются $\implies$ попарная независимость.

    $P(A \cap B \cap C) = \frac{1}{4} \not = \frac{1}{2} \cdot \frac{1}{2} \cdot \frac{1}{2} = P(A) \cdot P(B) \cdot P(C) \implies$ нет независимости в совокупности.
\end{example}

\begin{exerc}
    Д-ть, что $A_1, \dots A_m$ независимы в совокупности $\Leftrightarrow P(B_1 \cap B_2 \cap \dots \cap B_m) = P(B_1) \dots P(B_m)$, где $B_j = A_j$ или $\overline{A_j}$ (все $2^m$ равенств).
\end{exerc}

\begin{remark}
    Небольшое обобщение.

    $\Omega = \{ \omega_1, \dots, \omega_n \}$ -- пр-во элементарных исходов.

    Также у нас есть $p_1, \dots p_n: \ \sum_{i=1}^{n} p_i = 1, \ \forall i: \ p_i \geq 0$.

    $P(A) = \sum_{j: \ \omega_j \in A} p_j$.
\end{remark}

\begin{theorem}
    \textbf{Схема Бернулли}.

    орел $=$ успех $=$ 1.

    решка $=$ неудача $=$ 0.

    $P(\text{орел}) = p, \ 0 \leq p \leq 1$
    
    $P(\text{решка}) = 1 - p$

    Бросаем монету $n$ раз, получаем последовательность исходов:

    $\Omega = \{ x_1, x_2, \dots, x_n \}: \ x_j = 0 \text{ или } 1$.

    $\omega = (x_1, x_2, \dots, x_n), \ P(\{ \omega \}) = p^{\# i: \ x_i = 1} \cdot q^{\# i: \ x_i = 0} = p^{\sum x_i} \cdot q^{n - \sum x_i}$

    Хотим узнать:
    
    $P(\text{выпало ровно } k \text{ орлов}) = C_{n}^{k} p^k q^{n - k}$
    
    $P(i\text{-ое подбрасывание } = \text{ орел})$ -- независимые в совокупности по $i = 1, 2, \dots, n$.
\end{theorem}

\begin{theorem}
    \textbf{Полиномиальная схема}.

    $p_1, p_2, \dots, p_m: \sum p_i = 1$.

    $P(x_i = k) = p_k$, где $x_i \in \{ 1, 2, \dots, m \}$

    $\Omega = \{(x_1, x_2, \dots, x_n)\}, \ \omega = (x_1, x_2, \dots, x_n)$

    $P(\{ \omega \}) = p_1^{\# \{ i: x_i = 1 \}} \cdot \dots \cdot p_m^{\# \{ i: x_i = m \}}$

    $k_1 + k_2 + \dots + k_m = n$

    
    $P(k \text{ раз выпало 1, } k_2 \text{ раз выпало 2, ...}) = \binom{n}{k_1, k_2, \dots, k_m} p_1^{k_1} \cdot \dots \cdot p_m^{k_m} = \frac{n!}{k_1! \dots k_m!}$
\end{theorem}

\begin{theorem}
    \textbf{Эрдёша-Мозера}

    Рассмотрим турнир на $n$ команд. При каком наибольшем $k$ можно
    всегда выбрать команды $A_1, A_2 \ldots A_k$, так, что $A_i$
    выиграла у $A_j$, если $i < j$? При $k \leqslant 1 +  [2 \log_2 n] $
\end{theorem}

\begin{proof}
    Рассмотрим случайный турнир(Всего $\binom{n}{2}$, тогда Всего
    $2^{\binom{n}{2}}$ разных турниров. Случайный - берём из этой
    кучи наугад). TODO
    %$P(\text{A выиграла у B}) = \frac{1}{2}$. Рассмотрим $A_1, A_2,
    %\ldots A_k$ команды. $P(\text{A_1, A_2 \ldots A_k  подходят}) = (\frac{1}{2})^{\binom{k}{2}}$.
    %$P(\text{A_1, A_2 \ldots A_k можно переименовать, так, что они подошли}) \leqslant \frac{k!}{2^{\binom{k}{2}}}$
    %$P(\text{какие-то $k$ команд подошли}) \leqslant \binom{n}{k} \cdot \frac{k!}{2^{\binom{k}{2}}}$
    %Нужно понять, что если $k \geqslant 1 + \log_2 n$, то $\binom{n}{k} \frac{k!}{2^{\binom{k}{2}}} < 1$
    %$\binom{n}{k} \frac{k!}{2^{\binom{k}{2}}} = \frac{n(n - 1)(n - 2) \ldots (n - k + 1)}{2^{\frac{k(k - 1)}{2}}} < \frac{n^k}{(2^{\frac{k - 1}{2}})^k} = (\frac{n}{2^{\frac{k - 1}{2}}})^k$

    %Надо понять, что $2^{\frac{k - 1}{2}} \geqslant n \Leftrightarrow \frac{k - 1}{2} \geqslant \log_2 n$
\end{proof}

\Subsection{Предельные теоремы для схем Бернулли}

\begin{definition}
    Схема Бернулли с вероятностью успеха $p \in (0, 1)$. $S_n$ - число успехов при $n$ испытаниях. $P(S_n = k) = \binom{n}{k} p^k q^{n - k}$

    Что будет больше $P(S_{1000} = 220)$ при $p = \frac{1}{5}$ или $P(S_{2000}) = 360$ при $p = \frac{1}{6}$. Точные вычисления дают $0.008984$ и 
    $0.006625$ соответственно.

\end{definition}

\begin{theorem}
    \textbf{Пуассона}

    Схема Бернулли с $n$ испытаниями и вероятностью успеха $p_n$ - зависит от $n$.
    Если $np_n \rightarrow \lambda > 0$. Тогда $P(S_n = k) \rightarrow \frac{\lambda^k}{k!} e^{-\lambda}$
\end{theorem}

\begin{remark}
    Если $n p_n = \lambda$, то теорема верна при $k = o(\sqrt{n})$
\end{remark}

\begin{proof}
    $P(S_n = k) = \binom{n}{k} p^k (1 - p)^{n - k} = \frac{n(n - 1)\ldots(n - k + 1)}{k!} p_n^k (1 - p_n)^{n - k}$.
    $\sim \frac{n^k}{k!} p_n^k (1 - p)^{n - k} \sim \frac{\lambda^k}{k!} (1 - p_n)^n$. Прологарифмируем: $\ln (1 - p_n)^n = n \ln (1 - p_n)$
    $\sim -n p_n \sim -\lambda$

    Доказательство замечания:

    $\ln (1 - p_n)^k = k \ln (1 - p_n) \sim -kp_n \rightarrow 0$

    Осталось понять, что $\frac{n(n - 1) \ldots (n - k + 1)}{n^k} \to 1$ при $k = o(\sqrt{n})$.

    $\frac{n(n - 1) \ldots (n - k + 1)}{n^k} = 1 \cdot (1 - \frac{1}{n}) \ldots \cdot (1 - \frac{k - 1}{n}) \geqslant$
    $1 - \frac{1}{n} - \ldots -  \frac{k - 1}{n} = 1 - \frac{k(k - 1)}{2n} \rightarrow 1$

    Неравенство $(1 - x_1) \ldots (1 - x_k) \geqslant 1 - x_1 - x_2 - \ldots - x_k$ при $0 \leqslant x_i \leqslant 1$ - индукция.
\end{proof}

\begin{theorem}
    \textbf{Прохорова}

    Если $\lambda = np$, то $\sum_{i = 0}^{+\infty} |P(S_n = k) - \frac{\lambda^k}{k!}e^{-\lambda}| \leqslant \frac{2\lambda}{n} \cdot \min(2, \lambda)$

\end{theorem}

\begin{example}
    Игра в рулетку: 36 чисел и ноль. 

    $p = \frac{1}{37}, n = 111, np = 3 \lambda$.

    $P(S_111 = 3) = \binom{111}{3} (\frac{1}{37})^3 (1 - \frac{1}{37})^{111 - 3} = 0.227127$

    Из Пуассона $\frac{\lambda^3}{3!} e^{-\lambda} = 0.224$

    Видим, что приближение хорошее.

    $P(\text{выигрыш}) = 1 - P(S_{111} = 0) - P(S_{111} = 1) - P(S_{111} = 2) - P(S_{111} = 3) = $
    $1 - \frac{\lambda^0}{0!}e^{-\lambda} -$
    $\frac{\lambda^1}{1!}e^{-\lambda} - \frac{\lambda^2}{2!}e^{-\lambda} - \frac{\lambda^3}{3!}e^{-\lambda} = 0.352754$
    
    А по формулам $0.352768$
\end{example}

\begin{theorem}
    \textbf{Локальная предельная теорема Муавра-Лапласа}

    Схема Бернулии с вероятностью успеха $p \in (0, 1)$. $x = \frac{k - np}{\sqrt{npq}}$

    $P(S_n = k) \sim_{n \to +\infty} \frac{1}{\sqrt{2\pi n p q}} e^{\frac{-x^2}{2}}$

    Если $|x| \leqslant T$, то есть равномерность.
\end{theorem}

\begin{proof}
    $k = np + x\sqrt{npq} \geqslant np - T\sqrt{npq} \rightarrow +\infty$

    $n - k = nq - -x\sqrt{npq} \geqslant nq - T\sqrt{npq} \rightarrow +\infty$

    $P(S_n = k) = \binom{n}{k} p^k q^{n - k} = \frac{n!}{k!(n - k)!}p^k q^{n - k}$
    
    Напишем формулу Стирлинга:

    $\frac{n^n e^{-n} \sqrt{2\pi n}p^kq^{n-k}}{k^ke^{-k}\sqrt{2\pi n} (n-k)^{n - k} e^{-(n - k)} \sqrt{2 \pi (n - k)}} =$
    $\frac{p^k q^{n - k}}{(\frac{k}{n})^k \cdot (\frac{n - k}{n})^{n - k} \cdot \sqrt{2 \pi \frac{k}{n} (1 - \frac{k}{n})n}}$

    $\frac{k}{n} \rightarrow p, \frac{n - k}{n} \rightarrow q$

    Поэтому остаётся доказать, что $\frac{p^k q^{n - k}}{(\frac{k}{n})^k \cdot (\frac{n - k}{n})^{n - k}} \rightarrow e^{\frac{-x^2}{2}}$

    $k \ln \frac{k}{n} + (n - k) \ln \frac{n - k}{n} - k \ln p - (n - k)\ln q \rightarrow \frac{x^2}{2}$

    $\alpha = \frac{k}{n} \rightarrow p, \beta = \frac{n - k}{n} \rightarrow q$

    $n \alpha \ln \alpha + n \beta \ln \beta - n\alpha \ln p - n\beta \ln q \rightarrow \frac{x^2}{2} = n\alpha \ln \frac{\alpha}{p} + n \beta \ln \frac{\beta}{q}$

    Напишем Тейлора:

    $\frac{\alpha}{p} = 1 + x\sqrt{\frac{q}{p}} \frac{1}{\sqrt{n}}$

    $\frac{\beta}{q} = 1 - x \sqrt{\frac{p}{q}} \frac{1}{\sqrt{n}}$

    $\ln \frac{\alpha}{p} = \ln (1 + x\sqrt{\frac{q}{p}} \frac{1}{\sqrt{n}}) = x \sqrt{\frac{q}{p}} - \frac{1}{2}x^2\frac{q}{p} \frac{1}{n} + o(\frac{1}{n}) $ 

    $\ln \frac{\beta}{q} = \ln (1 - x \sqrt{\frac{p}{q} \frac{1}{\sqrt{n}}}) = -x \sqrt{\frac{p}{q}} \frac{1}{\sqrt{n}} - \frac{1}{2}x^2 \frac{p}{q} \frac{1}{n} + o(\frac{1}{n})$

    Сумма равна: $x\sqrt{pq}\sqrt{n} + x^2q - \frac{1}{2}x^2q + o(\frac{1}{n}) - x\sqrt{pq}\sqrt{n} + x^2p - \frac{1}{2}x^2p + o(\frac{1}{n}) = $
    $x^2 (\frac{q}{2} + \frac{p}{2}) + o(1) = \frac{x^2}{2} + o(1)$

    Извините, это было очень больно...
\end{proof}

\begin{remark}
    Если $\varphi (n) = o(n^\frac{2}{3})$ и $|k - np| \leqslant \varphi (n)$, то теорема тоже верна
\end{remark}

\begin{example}
    Всё та же рулетка. $n = 222, k = 111$. Пытаемся ставить на на четное/нечётное(кроме 0). $p = \frac{18}{37}$

    $P(S_{222} = 111) \approx \frac{1}{\sqrt{2 \pi n p q}} e^{\frac{-x^2}{2}} \approx 0.049395\ldots$

    Если считать точно, то получим $0.0493228\ldots$
\end{example}

\begin{theorem}
    \textbf{Интегральная теорема Муавра-Лапласа}

    $0 < p < 1$. $P(a < \frac{S_n - np}{\sqrt{npq}} \leqslant b) \rightarrow_{n \to \infty} \frac{1}{\sqrt{2\pi}} \int_a^b e^{\frac{-t^2}{2}} \, dt$

    Равномерно по $a$ и $b$
\end{theorem}







