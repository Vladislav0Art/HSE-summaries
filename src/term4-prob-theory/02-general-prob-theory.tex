\Subsection{Колмогоровская модель теории вероятности}

\begin{definition}
    $(\Omega, \mathcal{F}, P)$ - вероятностное пространство.

    $\Omega$ - множество или пространство элементарных исходов.

    $\mathcal{F}$ - $\sigma$-алгебра подмножеств $\Omega$. Элементы $\mathcal{F}$ - случайный события.

    $P$ - мера на $\mathcal{F}$ с условием $P(\Omega) = 1$.

    \begin{remark}
        Если $\Omega$ не более чем счётно, то можно взять $\mathcal{F} = 2^{\Omega}$
    \end{remark}
\end{definition}

\begin{definition}
    Условная вероятность. $A$ - событие, такое, что $P(A) > 0$.
    Тогда $P(B | A) = \frac{P(B \cap A)}{P(A)}$, где $A, B \in \mathcal{F}$.
\end{definition}

\begin{definition}
    Независимые события $A$ и $B$. Если $P(A \cap B) = P(A) \cdot P(B)$
\end{definition}

\begin{definition}
    Независимость в совокупности $A_1, A_2 \ldots A_n$. $P(A_{i_1} \cap \ldots \cap A_{i_k}) = P(A_{i_1}) \cdot \ldots \cdot P({A_{i_k}})$
    для всевозможных наборов индексов.
\end{definition}

\begin{definition}
    Последовательность событий $A_1, A_2 \ldots $ независимы - любой конечный набор событий
    независим в совокупности.
\end{definition}

\begin{lemma}
    \textbf{Бореля-Кантелли}

    $A_1, A_2, \ldots$ случайные события.

    \begin{enumerate}
        \item {
            Если $\sum_{n = 1}^{\infty} P(A_n) < +\infty$, то вероятность, что случилось бесконечное число из них равна 0.
        }
        \item {
            Если $A_1, A_2, \ldots$ независимы и $\sum_{n = 1}^{\infty} P(A_n) = +\infty$, тогда
            
            $P(\text{случилось бесконечное число из $A_n$}) = 1$.
        }
    \end{enumerate}
\end{lemma}

\begin{proof}
    $B = \bigcap_{n = 1}^{\infty} \bigcup_{k = n}^{\infty} A_k$.

    $\omega \in B \Longleftrightarrow \omega \in \bigcup_{k = n}^{\infty} A_k \ \forall n \Longleftrightarrow w \in A_k$ для бесконечного количества индексов $k$.

    Док-во этого факта:

    \begin{enumerate}
        \item $\Leftarrow:$ Лежит в каждом объединении, значит лежит в $B$.
        \item $\Rightarrow:$ $\omega$ лежит в пересечении. Пусть лежит в конечном - возьмём самый большой номер и получим противоречие.
    \end{enumerate}

    Док-во теоремы:

    \begin{enumerate}
        \item $P(B) = 0$ - хотим доказать.
        
        $B \subset \bigcup_{k = n}^{\infty} A_k \Rightarrow P(B) \leqslant P(\bigcup_{k = n}^{\infty} A_k) \leqslant \sum_{k = n}^{\infty} P(A_k)$, 
        а это хвост сходящегося ряда, а он стремится к нулю.

        \item Давайте смотреть на $\bar{A_1}, \bar{A_2}, \ldots$ - независимые события (следует из упражнения с прошлой лекции).
        
        $P(\bigcap_{k = 1}^n \bar{A_k}) = \prod_{k = 1}^{n} P(\bar{A_k}) \to_{n \to \infty} \prod_{k = 1}^{\infty}  P(\bar{A_k})$

        Но всё вложено по убыванию, по монотонности меры получаем $P(\bigcap_{k = 1}^{\infty} \bar{A_k}) = \prod_{k = 1}^{\infty} P(\bar{A_k}) = 
        \prod_{k=1}^{\infty} (1 - P(A_k))$

        Прологарифмируем это равенство.

        % $\ln (P(\bigcap_{n}^{\infty} \bar{A_k})) = \sum{k = n}^{\infty} \ln (1 - P(A_k)) \leqslant - \sum{k = n}^{\infty} P(A_k) = -\infty$ - сумма
        % хвоста расходящегося ряда.

        $\ln (P(\bigcap_{k = n}^{\infty} \bar{A}_k)) = \sum_{k = n}^{\infty} \ln(1 - P(A_k)) \leqslant \sum_{k = n}^{\infty} (-P(A_k)) = -\infty$ -- сумма хвоста расходящегося ряда.

        А значит мы логарифмировали $0 \Rightarrow P(\bigcap_{k = n}^{\infty} \bar{A_k}) = 0 \Rightarrow P(\bigcup_{n=1}^{\infty} \bigcap_{k = n}^{\infty} \bar{A}_k) = 0 \Rightarrow P(\bar{B}) = 0 \Rightarrow$
        
        $\Rightarrow \overline{\bigcup_{n=1}^{\infty} \bigcap_{k = n}^{\infty} \bar{A}_k} = \bigcap_{n=1}^{\infty} \bigcup_{k=n}^{\infty} A_k = B \Rightarrow P(B) = 1$.

        Добавим, что $B = \bigcap_{n = 1}^{\infty} B_n$, где $B_1 \supset B_2 \supset \ldots$ и $P(B) = \lim P(B_n) = 1$.
    \end{enumerate}
\end{proof}

\begin{theorem}
    \textbf{Закон нуля и единицы}

    Если $A_1, A_2 \ldots$ независимы, то $P(B) = 0$ или $P(B) = 1$.
\end{theorem}

\begin{example}
    Испытания Бернулли, успех с вероятностью $p$,
    
    $P(\text{ОРО встречается бесконечное число раз}) = \ ?$.

    $A_n = $ случилось $\text{ОРО}$ на позициях $n, n + 1, n + 2$.

    Тогда $A_1, A_4, A_7, \ldots$ независимы. $P(A_j) = pqp = p^2 q > 0$.

    Лемма Бореля-Кантелли говорит: бесконечное кол-во $A_{3k + 1}$ случится, если $\sum_{k=1}^{\infty} {P(A_{3k + 1})} = +\infty \implies P(\text{ОРО встречается бесконечное число раз}) = 1$. 
\end{example}